%%=============================================================================
%% Conclusie
%%=============================================================================

\chapter{Conclusie}%
\label{ch:conclusie}

% TODO: Trek een duidelijke conclusie, in de vorm van een antwoord op de
% onderzoeksvra(a)g(en). Wat was jouw bijdrage aan het onderzoeksdomein en
% hoe biedt dit meerwaarde aan het vakgebied/doelgroep? 
% Reflecteer kritisch over het resultaat. In Engelse teksten wordt deze sectie
% ``Discussion'' genoemd. Had je deze uitkomst verwacht? Zijn er zaken die nog
% niet duidelijk zijn?
% Heeft het onderzoek geleid tot nieuwe vragen die uitnodigen tot verder 
%onderzoek?

Het beveiligen van een Kubernetes-omgeving is een complexe taak die vraagt om het gebruik van diverse (open source) tools om de beveiligingskwaliteit te waarborgen. Bij het zoeken naar oplossingen voor beveiliging is het raadzaam te vertrouwen op de officiële documentatie, aangezien Kubernetes voortdurend evolueert en de beveiliging daarmee ook verandert. Door op de hoogte te blijven van recente ontwikkelingen en best practices binnen het uitgebreide Kubernetes-ecosysteem, kunnen organisaties hun beveiligingsstrategie optimaliseren en zich effectief verdedigen tegen nieuwe bedreigingen. De CVE database is een betrouwbare database van kwetsbaarheden die essentieel is voor het identificeren en aanpakken van beveiligingsrisico's. Tools zoals kube-linter en Trivy die zich baseren op de CVE database kunnen snel en efficiënt beveiligingsrisico's vinden, al blijven recente kwetsbaarheden gemeld in de CVE database een risico voor het systeem. Daarom is het belangrijk om als organisatie regelmatig updates en patches uit te voeren om de beveiliging van het systeem up-to-date te houden. Daarnaast is het essentieel om proactief te zijn en gebruik te maken van aanvullende beveiligingsmaatregelen, zoals scannen op malware, virussen of het monitoren en detecteren van het systeem bij onbekende toegang. Het beheren van beveiligingsrisico's is een continu proces dat aandacht en toewijding vereist om de integriteit en vertrouwelijkheid van systemen en gegevens te waarborgen. Zolang de tools onderhouden worden, blijven deze een cruciale rol hebben. Trivy is een must-have tool om de beveiliging te optimaliseren. Deze tool is ontworpen en onderhouden door Aqua Security, een betrouwbaar bedrijf dat zich focust op het verstrekken van beveiligingsoplossingen. Trivy detecteert en scant alles van misconfiguraties, tot sofware supply chain en container images. Deze tool heeft de mogelijkheid om misconfiguraties te scannen en hier oplossingen voor te geven, maar KubeLinter is gespecialiseerd in het ontdekken van misconfiguraties en sneller en efficiënter in het scannen van files en geeft een duidelijke weergave voor de misconfiguraties en de oplossingen. Zowel Kube-bench als Trivy zijn essentiële tools voor het scannen van een Kubernetes-cluster. Kube-bench biedt betere prestaties in vergelijking met Trivy en geeft ook duidelijkere informatie over de gedetecteerde kwetsbaarheden, inclusief mogelijke oplossingen. Daarnaast is OPA Gatekeeper een waardevolle tool om beleidsregels te implementeren en zo misconfiguraties en kwetsbaarheden te voorkomen bij het implementeren van Kubernetes-objecten. Uit dit onderzoek is echter gebleken dat de combinatie van tools, zoals Kube-bench, Trivy en kube-linter, de beste oplossing biedt voor een effectieve beveiliging van een Kubernetes-cluster. Door gebruik te maken van deze tools kunnen zowel de prestaties, gedetailleerde informatie over kwetsbaarheden als naleving van best practices worden verbeterd. Daarnaast kan het implementeren van OPA Gatekeeper als aanvullende tool bijdragen aan het handhaven van beleidsregels en het voorkomen van misconfiguraties en kwetsbaarheden tijdens de implementatie van Kubernetes-objecten. Het gebruik van deze tools in combinatie omvatten een breed scala aan tactieken en technieken die worden beschreven in de \textit{MITRE ATT\&CK} matrix.
