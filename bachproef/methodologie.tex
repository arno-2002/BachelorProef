%%=============================================================================
%% Methodologie
%%=============================================================================

\chapter{\IfLanguageName{dutch}{Methodologie}{Methodology}}%
\label{ch:methodologie}

\section{inleiding}

Dit onderzoek is ingedeeld in 2 delen:
\item theoretisch onderzoek
\item proof-of-concept

\section{Theoretisch onderzoek}

\begin{enumerate}
    \item Fase 1: Algemeen beeld vormen van de kubernetes structuur, componenten en belangrijke terminologie
    \item Fase 2: Onderzoeken van de kwetsbaarheden binnen een kubernetes cluster
    \item Fase 3: Onderzoeken van de tools en oplossingen voor de kwetsbaarheden van een kubernetes cluster
    \item Fase 4: Onderzoeken naar container en image scanning
\end{enumerate}

\subsection{Fase 1}
In de eerste fase was het doel om een beeld te vormen van hoe kubernetes in elkaar zit, welke componenten belangrijk zijn en de terminologie hierbij.
Dit maakt het makkelijker om volgende fases beter te kunnen verstaan. Deze informatie werd verzameld uit verschillende academische artikels, boeken en officiele websites.
De zoektermen die hiervoor zijn vastgelegd zijn: \textit{kubernetes, kubernetes cluster, docker}. 

\subsection{Fase 2}
In de tweede fase was het doel om de grootste kwetsbaarheden van kubernetes te kunnen onderzoeken. 
Uit de zoektermen van de vorige fase was er al reeds informatie verschaft over deze fase maar, de zoektermen werden hier nog meer uitgebreid: \textit{security, vulnerabilities}

\subsection{Fase 3}
De doelstelling uit de derde fase was het onderzoeken van de tools en oplossingen voor deze kwetsbaarheden. 
Uit de zoektermen van de vorige fase zijn er reeds al enkele oplossingen en tools gevonden maar, de zoektermen werden nog uitgebreid met: \textit{best practices, tools, guide}

\subsection{Fase 4}



\section{Proof-of-concept}
%% TODO: Hoe ben je te werk gegaan? Verdeel je onderzoek in grote fasen, en
%% licht in elke fase toe welke stappen je gevolgd hebt. Verantwoord waarom je
%% op deze manier te werk gegaan bent. Je moet kunnen aantonen dat je de best
%% mogelijke manier toegepast hebt om een antwoord te vinden op de
%% onderzoeksvraag.
