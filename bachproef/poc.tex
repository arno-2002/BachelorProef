%%=============================================================================
%% Proof-of-concept
%%=============================================================================

\chapter{\IfLanguageName{dutch}{Proof-of-concept}{Proof-of-concept}}%
\label{ch:Proof-of-concept}

\section{inleiding}

In dit hoofdstuk worden bepaalde technieken, tools of oplossingen, beschreven in de literatuurstudie, die helpen bij het implementeren van de beveiliging van een kubernetes toegepast met behulp van minikube en kubernetes goat.

\section{Kubernetes Goat}

\textit{Kubernetes Goat is een interactieve leeromgeving voor Kubernetes-beveiliging. Het heeft opzettelijk kwetsbare ontwerpscenario's om de veel voorkomende misconfiguraties, echte kwetsbaarheden en beveiligingsproblemen in Kubernetes-clusters, containers en cloud native-omgevingen te laten zien. \autocite{kubernetesGoat-2023}

Minikube is een hulpmiddel dat het toelaat om Kubernetes lokaal te draaien. Minikube draait een all-in-one of een multi-node lokale kubernetes cluster op een persoonlijke computer waar Windows, macOS of linux als besturingssysteem \autocite{KubernetesDocs-2023}. Als proof-of-concept wordt er een minikube opgestart op een Linux machine met Docker als driver. In deze Kubernetes cluster wordt er een voorbeeldapplicatie gemaakt om de gevonden tools en oplossingen in uit te werken. 

\begin{enumerate}
    \item Minikube installeren
    \item Minikube starten
\end{enumerate}

