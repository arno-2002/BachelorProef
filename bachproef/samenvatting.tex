%%=============================================================================
%% Samenvatting
%%=============================================================================

% TODO: De "abstract" of samenvatting is een kernachtige (~ 1 blz. voor een
% thesis) synthese van het document.
%
% Een goede abstract biedt een kernachtig antwoord op volgende vragen:
%
% 1. Waarover gaat de bachelorproef?
% 2. Waarom heb je er over geschreven?
% 3. Hoe heb je het onderzoek uitgevoerd?
% 4. Wat waren de resultaten? Wat blijkt uit je onderzoek?
% 5. Wat betekenen je resultaten? Wat is de relevantie voor het werkveld?
%
% Daarom bestaat een abstract uit volgende componenten:
%
% - inleiding + kaderen thema
% - probleemstelling
% - (centrale) onderzoeksvraag
% - onderzoeksdoelstelling
% - methodologie
% - resultaten (beperk tot de belangrijkste, relevant voor de onderzoeksvraag)
% - conclusies, aanbevelingen, beperkingen
%
% LET OP! Een samenvatting is GEEN voorwoord!

%%---------- Nederlandse samenvatting -----------------------------------------
%
% TODO: Als je je bachelorproef in het Engels schrijft, moet je eerst een
% Nederlandse samenvatting invoegen. Haal daarvoor onderstaande code uit
% commentaar.
% Wie zijn bachelorproef in het Nederlands schrijft, kan dit negeren, de inhoud
% wordt niet in het document ingevoegd.

\IfLanguageName{english}{%
\selectlanguage{dutch}
\chapter*{Samenvatting}
\lipsum[1-4]
\selectlanguage{english}
}{}

%%---------- Samenvatting -----------------------------------------------------
% De samenvatting in de hoofdtaal van het document

\chapter*{\IfLanguageName{dutch}{Samenvatting}{Abstract}}

Kubernetes is een bekend platform voor het orkestreren van containers en heeft een grote impact op de IT. Naarmate het gebruik van Kubernetes toeneemt, groeit ook de behoefte aan beveiliging. Securex, een sociaalzekerheidskantoor gevestigd in Gent, heeft de noodzaak om de beveiliging van hun Kubernetes-clusters te optimaliseren. Dit onderzoek richt zich op het vinden van tools en oplossingen om de kwetsbaarheden en beveiligingsrisico's te beperken. Eerst wordt onderzocht wat de kwetsbaarheden van Kubernetes zijn, gevolgd door een analyse van mogelijke oplossingen voor deze kwetsbaarheden. Nadat de kwetsbaarheden en oplossingen zijn onderzocht, wordt er een proof-of-concept ontwikkeld met een WordPress-MySQL-applicatie om verschillende tools en oplossingen te testen. De onderzoeksvraag luidt als volgt: Hoe kunnen open-source tools en oplossingen beschikbaar gesteld worden om zowel de geïnstalleerde software op Kubernetes te scannen als de configuratie van Kubernetes te valideren? \\

In het onderzoek is er aandacht besteed aan het identificeren en begrijpen van de specifieke kwetsbaarheden en beveiligingsrisico's die verband houden met Kubernetes. Dit omvat bijvoorbeeld het blootstellen van onbeveiligde API's, verkeerde configuratie van toegangsrechten en mogelijke beveiligingslekken in de gebruikte containers. Er worden verschillende tools genoemd, zoals Kube-bench, Kube-linter, OPA, Kube-hunter en Trivy. De keuze voor deze tools is gebaseerd op een eerdere studie van Red Hat, waarin werd onderzocht welke tools het populairst zijn voor het beveiligen van Kubernetes. Uit dit onderzoek is gebleken dat de combinatie van deze tools essentieel is voor de beste beveiliging. Uit dit onderzoek is ook gebleken dat het van cruciaal belang is om de beveiliging van Kubernetes-clusters regelmatig te monitoren en bij te werken, omdat het beveiligingslandschap voortdurend evolueert. Dit omvat het volgen van nieuwe beveiligingsupdates, het implementeren van best practices en het continu evalueren van de effectiviteit van de gebruikte tools en oplossingen.
