\chapter{\IfLanguageName{dutch}{Stand van zaken}{State of the art}}%
\label{ch:stand-van-zaken}

% Tip: Begin elk hoofdstuk met een paragraaf inleiding die beschrijft hoe
% dit hoofdstuk past binnen het geheel van de bachelorproef. Geef in het
% bijzonder aan wat de link is met het vorige en volgende hoofdstuk.

% Pas na deze inleidende paragraaf komt de eerste sectiehoofding.
\section{Definities}

\begin{itemize}
    \item \textit{Container: Een container is een standaard software-eenheid die code en al zijn afhankelijkheden verpakt, zodat de toepassing snel en betrouwbaar van de ene computeromgeving naar de andere draait \autocite{Docker-2023}. Containers zijn lichtgewicht en bevatten alles wat nodig is om de applicatie te draaien, dus u hoeft niet te vertrouwen op wat er momenteel op de host is geïnstalleerd. \autocite{DockerDocs-2023} }
    \item \textit{Docker: Docker biedt de mogelijkheid om een aplicatie te verpakken en uit te voeren in een geïsoleerde omgeving, genaamd een container.  \autocite{DockerDocs-2023} }
\end{itemize}

\ref{fig:KubernetesContainers} Geeft een beter beeld wat containers zijn en welke functie docker heeft.

\begin{figure}[h]
    \includegraphics[width=.49\textwidth]{graphics/container.png}
    \caption{\label{fig:KubernetesContainers}Container architectuur \autocite{Docker-2023}}
\end{figure} 

\section{Kubernetes}

Kubernetes, ook wel afgekort als K8s, is een gratis en open-source systeem dat wordt gebruikt om de implementatie, schaling en het beheer van containerapplicaties te automatiseren. Het biedt de mogelijkheid om containers die deel uitmaken van een applicatie te groeperen en te beheren als logische eenheden, waardoor het gemakkelijk wordt om deze te ontdekken en te beheren \autocite{KubernetesDocs-2023}.


. Deze groep noden bestaan uit minstens één master node en een aantal worker nodes. Binnen deze worker nodes zijn er pods waarin containers draaien. Deze containers bevatten applicaties zoals een SQL database of een website die draait met nginx. In de literatuurstudie wordt dit uitgebreid besproken. Als er applicaties binnen deze cluster in containers worden gedraaid moet er een zekerheid zijn dat deze containers goed beveiligd zijn van eventuele kwetsbaarheden. 


Dit hoofdstuk bevat je literatuurstudie. De inhoud gaat verder op de inleiding, maar zal het onderwerp van de bachelorproef *diepgaand* uitspitten. De bedoeling is dat de lezer na lezing van dit hoofdstuk helemaal op de hoogte is van de huidige stand van zaken (state-of-the-art) in het onderzoeksdomein. Iemand die niet vertrouwd is met het onderwerp, weet nu voldoende om de rest van het verhaal te kunnen volgen, zonder dat die er nog andere informatie moet over opzoeken \autocite{Pollefliet2011}. Bie ba bla

Je verwijst bij elke bewering die je doet, vakterm die je introduceert, enz.\ naar je bronnen. In \LaTeX{} kan dat met het commando \texttt{$\backslash${textcite\{\}}} of \texttt{$\backslash${autocite\{\}}}. Als argument van het commando geef je de ``sleutel'' van een ``record'' in een bibliografische databank in het Bib\LaTeX{}-formaat (een tekstbestand). Als je expliciet naar de auteur verwijst in de zin, gebruik je \texttt{$\backslash${}textcite\{\}}.
Soms wil je de auteur niet expliciet vernoemen, dan gebruik je \texttt{$\backslash${}autocite\{\}}. In de volgende paragraaf een voorbeeld van elk.

\textcite{Knuth1998} schreef een van de standaardwerken over sorteer- en zoekalgoritmen. Experten zijn het erover eens dat cloud computing een interessante opportuniteit vormen, zowel voor gebruikers als voor dienstverleners op vlak van informatietechnologie~\autocite{Creeger2009}.

\lipsum[7-20]

